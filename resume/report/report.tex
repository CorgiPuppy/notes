\documentclass[a4paper,11pt]{article}

\usepackage[utf8]{inputenc}
\usepackage[russian]{babel}

\usepackage{geometry}
\usepackage{enumitem}
\usepackage{hyperref}
\usepackage{titlesec}

\geometry{margin=1in}

\titleformat{\section}{\large\bfseries}{\thesection}{1em}{}
\titleformat{\subsection}{\normalsize\bfseries}{\thesubsection}{1em}{}

\usepackage{times}

\begin{document}

	\begin{center}
		{\Large\bfseries Андрей Золотухин} \\
		\vspace{0.2cm}
		Студент РХТУ им. Менделеева, направление «Информационные системы и технологии» \\
		\href{mailto:andrey.zolotukhin2017@mail.ru}{andrey.zolotukhin2017@mail.ru} | +7 (982) 147-98-31 | \href{https://github.com/CorgiPuppy}{github.com/CorgiPuppy}
	\end{center}

	\section{Цель}
	Студент-программист, в поиске практической деятельности в Институте программных систем им. А.К. Айламазяна РАН, где смогу применять навыки разработки на C++, Python, веб-технологиях, администрировании систем или компьютерном зрении для решения реальных задач и развития в области информационных систем.

	% Skills section
	\section{Навыки}
	\begin{itemize}[leftmargin=*]
		\item \textbf{Языки программирования}: C/C++, Java, Python, JavaScript, PHP, Bash, Lua, Go, Zig, Ruby, Rust, Lisp
		\item \textbf{Веб-разработка}: HTML, CSS, LESS, PHP (сессии, авторизация), JavaScript (таймеры, интерактивные элементы)
		\item \textbf{Базы данных}: PostgreSQL (\href{https://github.com/CorgiPuppy/data-mgmt-labs/tree/master/lab6}{GitHub}), MongoDB (\href{https://github.com/CorgiPuppy/data-mgmt-labs/tree/master/lab8}{GitHub})
		\item \textbf{Алгоритмы и структуры данных}: сортировки, списки, графы, деревья, хэш-таблицы, динамическое программирование (\href{https://github.com/CorgiPuppy/algo-ds-labs}{GitHub})
		\item \textbf{Системное администрирование}: настройка Arch Linux, Bash-скрипты для автоматизации, работа с процессами и сигналами
		\item \textbf{Другое}: LaTeX, Typst, Makefiles, компьютерное зрение (\href{https://roboflow.com/}{Roboflow}), моделирование физико-химических систем
	\end{itemize}

	% Projects section
	\section{Проекты}
	\begin{itemize}[leftmargin=*]
		\item \textbf{Система управления библиотекой (C++)}: Разработал приложение для анализа текстов произведений, реализующее классы библиотеки, книг и посетителей. Реализовал функционал поиска книг по жанру, объему и возрастным ограничениям, а также моделирование работы библиотеки на 30 дней с выводом статистики (средние оценки, время чтения). \href{https://github.com/CorgiPuppy/cpp-labs/tree/master/lab8}{GitHub}
		\item \textbf{Эмуляция файловой системы (C++)}: Создал программу для управления каталогом файлов с поддержкой команд создания, удаления, переименования и поиска по шаблонам (*, ?, |). Реализовал обработку исключений для некорректных операций. \href{https://github.com/CorgiPuppy/cpp-labs/tree/master/lab10}{GitHub}
		\item \textbf{Обучающий сайт по сборке кубика Рубика (PHP, HTML, CSS, JavaScript)}: Разработал веб-приложение с авторизацией, сессиями и таймером для обучения сборке кубика Рубика. Использовал LESS для стилизации. \href{https://github.com/CorgiPuppy/web-labs/tree/master/myOwnSite}{GitHub}
		\item \textbf{Моделирование физико-химических систем (Python, Rust)}: Написал программы для расчета равновесных концентраций и решения дифференциальных уравнений методом Рунге-Кутты 7-го порядка (\href{https://github.com/CorgiPuppy/phys-chem-labs/tree/master/lab7}{GitHub}). Реализовал модель Вильсона для бинарных парожидкостных систем (\href{https://github.com/CorgiPuppy/phys-chem-labs/tree/master/lab3}{GitHub}).
		\item \textbf{Скрипты для DevOps-задач (Bash, AWK)}: Разработал Bash-скрипты для игры «Похитители и Патрульные», реализующие процессы и сигналы для защиты/чтения файла (\href{https://github.com/CorgiPuppy/info-sys-admin-labs/tree/master/lab8}{GitHub}). Создал калькулятор на AWK и Bash для автоматизации вычислений (\href{https://github.com/CorgiPuppy/info-sys-admin-labs/tree/master/lab7/src/task3}{GitHub}).
		\item \textbf{Обработка больших данных (Python, Roboflow)}: Реализовал метод автоматического обнаружения и сегментации круглых частиц на микроскопических изображениях с использованием компьютерного зрения. \href{https://github.com/CorgiPuppy/big-data-labs/tree/master/labParticles}{GitHub}
	\end{itemize}

	% Experience section
	\section{Опыт}
	\begin{itemize}[leftmargin=*]
		\item \textbf{Учебная практика, РХТУ им. Менделеева (2024)}: Настраивал Arch Linux на виртуальной машине, изучал сетевые технологии (обжимка кабеля). \href{https://github.com/CorgiPuppy/intro-edu-practice}{GitHub}
	\end{itemize}

	% Education section
	\section{Образование}
	\begin{itemize}[leftmargin=*]
		\item \textbf{РХТУ им. Менделеева}, направление «Информационные системы и технологии», 2022–н.в.
	\end{itemize}

	% Interests section
	\section{Интересы}
	Компьютерное зрение, машинное обучение, разработка мобильных приложений под Android, системное администрирование, автоматизация процессов, моделирование физико-химических систем.

\end{document}
